% -*- coding: utf-8 -*-
\chapter*{はじめに}
\section*{錯視図形とは}
\cite[序 \romannumeral1ページ]{OIHandbook}によれば錯視図形とは
\begin{quotation}
 「知覚された外界の対象(形や色など)が、実際の物理的な構造と異なることを
 知って驚き、それらに興味を抱く現象である」と要約できるように思われる
\end{quotation}
と書かれています。同書の口絵には多数の錯視図形が掲げられています。

図\ref{vint}や\ref{zollner}のような
古典的な錯視図形は幾何学的な模様の繰り返しになっています。

\renewcommand{\thefigure}{\arabic{figure}}
\begin{center}
 \ShowGraphicP{0.6}{ht}{vint}{\OIIdxM{ヴィント}}{vint}
\end{center}
 \begin{center}
 \ShowGraphicP{0.6}{ht}{zollner}{\OIIdxM{ツェルナー}}{zollner}
 \end{center}

この講座では古典的なものから最近作成された錯視図形のうちいくつか題材を選
んで実際に作図してみることにしています。
\section*{錯視図形の作図方法}
図\ref{vint}や\ref{zollner}は \Opera というブラウザーの画面です。
これらの図は Photoshop や Illustrator に代表される絵を描くためのソフ
トウェアを用いて描いたものではありません。その理由は次のとおりです。
\begin{itemize}
 \item 幾何学的な模様を繰り返して同じように描くには手作業では大変に
       なります。たとえば図\ref{zollner}において斜めに交わった直線の角度
       を変える場合を考えてみてください。もう一度はじめから描きなおす必
       要があります。同様に、
       直線の間隔や色、線の幅を少し変えたりして図形の見え方がどのように
       変わるかをいろいろ試すには時間がかかってしまいます。
 \item 図形の一部を移動させたりするアニメーションを付けることができませ
       ん。
 \item 実際に完成した図形の特定の部分の情報を直接確かめる方法がありませ
       ん。
\end{itemize}
これらの条件は Flash を用いれば避けることができると思われるかも知れませ
ん。しかし、次の理由で今回は Flash を用いて描くことはしません。
\begin{itemize}
 \item 線を描いたあとにそれぞれにアニメーションを付け加えるためにはぞれ
       ぞれの線分をムービークリップに変換して個々にアニメーションをつけ
       る必要があります。ひとつ作成してコピーをしたとしても長さを変更す
       るときははじめからやり直しです。
 \item  Flash には ActionScript というプ
       ログラム言語を用いることでいろいろな処理ができます。残念ながら
       ActionScript でプログラムするための解説書は余り多くありません。
 \item Adobe が販売する Flash を作成するツールはそんなに安いソフト
       ウェアではありません。
 \item Flash ムービーになったファイルのソースコードを直接見ること
       ができません。
\end{itemize}
上で述べた点を克服するために
%図形の色や線の幅などの要素を変えて
%錯視の見える効果がどのように変化するかを確かめるために
この講座ではW3Cが定めた SVG(Scalabe Vector Graphics)という規格で図形を記
述することにします。W3CとSVGについては本文を参照してください。
\iffalse\footnote{図形をプログラムで記述するのであれば
PostScript という選択もあります。残念ながら PostScript はグラディエー
ションやアニメーションを利用できないので今回の候補からはずしました。
このテキストの表示のフィックの渦巻き錯視は PostScript を用いて描いていま
す。}
\fi
\iffalse
なお、\keyitem{W3C}はWebで利用される技術の規格を規定している団体です(図
\ref{W3C}参照)。ホームページを記述する HTML の規格もここで制定されていま
す。

\ShowGraphicP{0.4}{p}{w3c}{W3Cのホームページ}{W3C}

SVG は次のような特徴があります。

\paragraph{ベクトル方式による図形の表現}
       SVG で描かれた図形はいくら拡大してもドットが見えることはありませ
       ん。これはその名前にあるようにベクトル方式\footnote{図形を点の集
       まりとして表示する(ビットマップ方式)のではなく、形やその性質を与
       えることで記述する方法です。Adobe の Photoshop や Windows につい
       てくるペイントはビットマップ方式の図を作成します。Adobe の
       Illustlator はベクトル方式で右傾を保存できます。}で図形を記述する
       からです。
 \paragraph{XML に基づく規格}
       SVG は XML(eXtensible Markup Language) という規格に基づいて図形の
       記述する方法が定義されています。
       つまり、図形はすべてテキストとして定義されます。SVG のファイルは
       メモ帳などの簡単なエディタで作成することができます。
 \paragraph{アニメーションのサポート}Flash と同等のアニメーション
       が使用できます。
 \paragraph{DOM のサポート}プログラムから SVG のなかにある図形の
       形を変えたり、追加する方法が規定されています。
 \paragraph{イベント処理}マウスのクリックなどを処理する方法が定義
       されています。
 \paragraph{オープンな規格}
       SVG は W3C という非営利団体が規定するオープンな規格です。利用する
       ために料金を支払う必要はありません。
\fi

それでは、SVG を通じて錯視図形の世界を楽しみましょう。
\renewcommand{\thefigure}{\thechapter.\arabic{figure}}
