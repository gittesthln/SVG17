% -*- coding: utf-8 -*-
\chapter{\JS を利用したSVG図形の生成}\label{MakeInterractiveSVG}
この章ではSVGの要素をプログラムから制御する方法を学びます。
これを利用するとHTML上で入力されたデータやユーザーの動作に反応して
変化するSVGの図形を作成したり、このテキストの表紙にあるような複雑な図形
をプログラムで描くことができるようになります。

なお、この章でにおけるプログラミングスタイルが章の始めと終わりの方で異なっ
ています。\JS でより良い
プログラミングをするためには気を付けなければならないことが多くあります。
当初は理解しやすくするために簡単な記述をしています。本格的
なプログラムを組むために注意する点については後の方で解説をしますので、そ
のプログラミングスタイルに慣れるようにしてください。
\section{インターラクティブなSVGを作成するための準備}
\iffalse
前節でのsvgファイルの作成は与えられたパラメータを与えるとそれに応じたも
のを返すというサーバーとのやり取りで行っていました。ちょっとの値の変化で
もサーバーと更新するのでネットワークの負荷がかかりすぎるという欠点があり
ます。そこでクライアント側の操作を認識してそれに応じてsvgファイルの中を
書き換える手段があれば操作性が向上します。
この手段を実現するために必要な
ものが
\fi
マウスのクリックなどに反応して図形を変えることを実現するためには
プログラミングとSVGの図形がどのように内部で扱われているかを
知っておく必要があります。

利用できるプログラミング言語は {\JS} です。SVGの図形(XML の構造)
を内部で管理するためには DOM の概念が必要です。
\input CH7/0701Javascript.tex
\input CH7/0702DOM.tex
\iffalse
\input CH7/0710Event.tex
\input CH7/0720JS-in-svg.tex
%%%%\input CH7/0730KeyPress.tex
\input CH7/0740MakeObjectsAtStart.tex
%%%%\input 071-03DOMtree.tex
\input CH7/0750SelfAnimation.tex
\input CH7/0760HTML.tex
%
\input CH7/0770Closure.tex
%%%%\input CH7/0720SVGApplication.tex
%%%%\input 071-10Ajax.tex
\fi