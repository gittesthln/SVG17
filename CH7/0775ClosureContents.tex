\subsubsection{変数のスコープと仮引数への値の渡し方}
\ElmJ{変数のスコープ}とは使用する変数がどの範囲で参照できるか概念です。
\begin{itemize}
 \item \JS は変数は宣言しなくても使用可能です。最近の\JS の仕様では宣言
			 しない変数は使えないモードを宣言できます。
 \item 関数などの外で宣言されたり、宣言がない変数はどこからでも参照が可
			 能な変数(\ElmJ{グローバル変数})になります。
 \item 関数内で宣言された変数はそれ以下の関数内で参照可能です。
 \item \ElmJ{var}で宣言した変数は関数内の先頭で宣言したものとして扱われ
			 ます。
 \item 新しいEcmaScriptでは定義されたブロック内で有効な\ElmJ{let}と
			 \ElmJ{const}による変数の宣言が可能となっています。
\end{itemize}
関数の引数に対して、単純な変数は値渡し、配列は参照渡しとなります。
\begin{Verbatim}[numbers=left, fontsize=\small,
	commandchars=\\//,codes={\catcode`$=3\catcode`^=7}]
>foo =function(X,Y,Z) {
    X = X*2;
    Y[0] *= 2;
    Z.x *=2;}
function (X,Y,Z) {
    X = X*2;
    Y[0] *= 2;
    Z.x *=2;}
>A=1
1
>B=[2,3]
$\blacktriangleright$[2, 3]
>C={x:10,y:20};
$\blacktriangleright$Object {x: 10, y: 20}
>foo(A,B,C);
undefined
>A;
1
>B;
$\blacktriangleright$[4, 3]
>C;
$\blacktriangleright$Object {x: 20, y: 20}
\end{Verbatim}
次に\ElmJ{let}と\ElmJ{var}による宣言で変数のふるまいの違いを見てみます。

次のリストは3つの関数を配列に格納して戻す関数をその場で実行しています。
\begin{Verbatim}[numbers=left]
let funcs = (function() {
  f = [];
  let i;
  for(i=0;i<3;i++) {
    f[i] = function(){console.log(i);};
  }
  return f;
  })();
\end{Verbatim}
4行目から6行目のループの引数の値を返す簡単な関数です。
\begin{Verbatim}
>funcs[0];
function (){console.log(i);}
>funcs[0]();
3
undefined
\end{Verbatim}
0番目の関数を実行すれば0がコンソールに出力されません。この関数
が実行されたときは4行目からのループが終了しているので、\texttt{i}の値は3になって
いるからです。0を返すためには3行目の\texttt{i}の宣言
を取り除き、\ElmJ{for}のなかで変数\texttt{i}を\ElmJ{let}で宣言すれば実現
できます。

なお、変数の宣言を\ElmJ{var}ですると、この不具合は解消されません。
\ElmJ{for}内で\ElmJ{var}で宣言された変数が関数内部の先頭で宣言されていた
ものとみなされるからです。解消す
るためにはテクニックが必要となりますが、\ElmJ{let} で解消できるのでここ
では解説しません。
\subsubsection{クロージャ}
いままでの\JS のプログラムでは、他の関数とデータのやり取りにグローバル変
数を多用してきました。グローバル変数を多用することは、複数人でプログラム
を開発するときなどに変数名の衝突がおき、それによってバグが発生する可能性
が増大します。これを避ける方法の一つとして%前節で示したような
関数の中で別の関数オブジェクトを作成する方法があります。
関数の中で新しく関数を定義すると、新しい関数ではその関数を定義する関数内
のローカル変数を参照できます。関数が実行されるときの環境を含めた状態を
\keyitem{クロージャ}と呼びます。

リスト\ref{svg-cycloid-animation-with-closure}はリスト
\ref{svg-cycloid-animation}(\pageref{svg-cycloid-animation}ページ)を改造
して、グローバル変数をまったく使わない
サイクロイドをアニメーションで描くものです。
\SVGListN{サイクロイドを描く --- アニメーション版(グローバル変数なし)}
{svg-cycloid-animation-with-closure3}
{svg-cycloid-animation-with-closure}
\begin{itemize}
 \item すべてのグローバル変数を\Event{onload}後に実行する関数内に移動しています
       (\LineR{GlobalS}{GlobalE})。
 \item その関数内でロカールな関数を定義しています
			 (\LineR{ClosureS}{ClosureE})。
\begin{itemize}
 \item 関数名は\Func{DrawNext}となっています(\Line{ClosureS})。
 \item \ElmJ{function}の前に\texttt{(}がついています。これに対応する
			 \texttt{)}は\Line{ClosureE}にあります。
			 %
  そのあとにある\texttt{()}はこの関数の引数リストです(ここでは
			 引数がないので空です)。
			 
  これにより関数を定義してその場で実行できます。
 \item 関数の定義位置が変わっただけで、残りのコードはほとんど同じです。
 \item 一番の違いは\Attrib{d}で必要となる座標データを配列として保存している
			 ことです(\Line{GlobalE})。
 \item また、\Line{push}で新たに計算した座標をこの配列に追加しています
			 (\ElmJ{push}メソッド)。
 \item \Line{join}の\ElmJ{join}は配列の要素を、引数の文字列をはさんでつ
       なげるメソッドです。配列内の文字列の連接
       を連接演算子\texttt{+}を用いるよりも簡単に記述できます。
 \item \Line{callee}ではこの関数自身がこの後、
			 呼び出せるように設定しています\footnote{関数の引数のリストを参照
			 する\ElmJ{arguments}のメンバー\ElmJ{callee}を用いると、無名関数を
			 参照できます。最新版の\JS では非推奨となっているので無名関数にし
			 ません。}。
\end{itemize}

\end{itemize}
