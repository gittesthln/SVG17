Web ページ上で作成したデータをそのままブラウザが動作しているコンピュータ
上に保存することはセキュリティ上できません。しかし、ブラウザが管理する領
域に保存することは可能です。この節ではそれらの方法につい
て簡単な解説をします。

\ElmJ{Web Storage}とは HTML5 で導入された、Webページ間のデータの共有する仕組みで
す。HTML5以前ではクッキーと呼ばれるデータをクライアントとサーバー
の間で共有してサーバー側からデータの共有をサポートする仕組みがあります。
クッキーでは小さなデータしか扱えず、サーバーとクライアントとのやりがある
たびにやり取りを行うのでセキュリティや速度の面で問題がありました。

HTML5ではブラウザのあるページ以降のページにだけ
存在する\ElmJ{sessionStorage}とページが閉じられてもデータが存続できる
\ElmJ{localStorage}があります。これらの値はすべて文字列で
しか保存できません。

リスト\ref{pinna-storage.js}はリスト
\ref{pinna.html}(\pageref{pinna.html}ページ)
において最後に設定した色をStorageに保存しておくように\JS ファイルを書き
直したものです。
\JSListN{Storageを利用する}{pinna-storage}{pinna-storage.js}
このリストは\texttt{Storage}に対応する部分が追加されているだけです。
\begin{itemize}
 \item \Line{storage}で変数\texttt{Storage}に
       \texttt{window.localStorage}を代入しています。これは後で
       \ElmJ{sessionStorage}に書き直すのを簡単にするためです。
 \item \Line{setFromStorage1}ではテキストボックス「色1」の初期値を
       \texttt{Storage.C1}が定義されていたらその値を、そうでない場合には
       \texttt{"red"}に設定しています。
 \item 「色2」の方も同様です。
 \item 「設定」ボタンが押されたときに呼び出される関数\texttt{DrawFigs}で
       は色の情報を対応する\texttt{Storage}に格納しています
       (\Lines{saveFromStorage1}{saveFromStorage2})。
\end{itemize}
図\ref{pinna-storage.html}はデベロッパーツールで\ElmJ{localStorage}の
値の確認をしているところです。
\ShowFig{0.8}{ht}{pinna-storage}
{\protect\texttt{sessionStorage}の値の確認}{pinna-storage.html}
\begin{Problem}\upshape
 リスト\ref{pinna-storage.js}について次のことを確認しなさい。
 \begin{enumerate}
  \item \ElmJ{localStorage}の値上でダブルクリックする(または右クリック)
        と値が直接書き直せたり、キーの削除ができます。
        値を書き直した後で開きなおすとその値
        が反映された図が描けます。
%  \item これらのファイルを別のフォルダにコピーして表示すると、初期値で表
%        示される。
  \item リストの\Line{storage}をコメントアウトして、\Line{storage2}のコ
        メントを外して、別の値を設定する。
        \begin{itemize}
         \item \ElmJ{locaStorage}の方ではな
        く、\ElmJ{sessionStorage}の方が書き直されている。
         \item 再表示すると最後に設定した値が反映されない。
         \item 再表示後は\ElmJ{sessionStorage}の値が全くない。
        \end{itemize}
\end{enumerate}
\end{Problem}

次のリストは配列のメソッドやWebStorageに配列のデータを格納するように変更
したものです。リスト\ref{pinna-storage.js}と異なるところだけ解説します。
\JSListN{Storageを利用する}{pinna-storage-rev}{pinna-storage-rev.js}
\begin{itemize}
 \item \Line{selectorAll}では\DOMM{querySelectorAll}を用いて、色が設定
       されている\ElmH{input}のリストを得ています。
       \DOMM{querySelectorAll}ではCSSのセレクタで該当する要素をすべて得
       ることができます。ここでは\ElmH{input}で\AttribH{type}が
       \AttribHVal{text}のものを指定しています。\HTML 内での書き方と同じ
       にするので\Verb+"+を記述するために\Verb+\"+と記述しています。%"
 \item \Line{getDataFromStorage}では\Verb+Storage.C+に値が設定されていれ
       ばその値を\ElmJ{JSON}の\JSM{parse}メソッドを用いて\JS のオブジェ
       クトに変更しています。ない場合は初期値の配列にしています。
 \item \Line{setValuesInput}では、\Line{selectorAll}で得られた
       \ElmH{input}の配列に値を設定しています。
 \item 同様に、\Line{getDataFromInputs}では配列のメソッド\ElmJA{map}を
       利用するために\JSM{call}を用いて\ElmH{input}のデータを配列に代入
       しています。
 \item \Line{saveDataInStorage}では\Verb+Storage.C+に保存するために
       \ElmJ{JSON}の\JSM{stringify}メソッドを使用しています。
\end{itemize}
\begin{Problem}
 \upshape
 リスト\ref{pinna-storage-rev.js}について次の事項に答えなさい。
 \begin{enumerate}
  \item \Line{selectorAll}のCSSセレクタの記述で\Verb+\"+のエスケープを使わない
        記述があるか。
  \item \Line{saveDataInStorage}で\ElmJ{JSON}の\JSM{stringify}メソッドを
        使用しないでそのまま代入するとどうなるか。
 \end{enumerate}
\end{Problem}