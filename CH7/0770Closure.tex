% -*- coding: utf-8 -*-
\section{より進んだ \JS プログラミング}
\subsection{配列のメソッドを使う}
\input JavaScript/2005ArrayMethod.tex
\subsection{クロージャの利用によるグローバル変数の個数の減少}
\subsubsection{変数のスコープと仮引数への値の渡し方}
\ElmJ{変数のスコープ}とは使用する変数がどの範囲で参照できるか概念です。
\begin{itemize}
 \item \JS は変数は宣言しなくても使用可能です。最近の\JS の仕様では宣言
			 しない変数は使えないモードを宣言できます。
 \item 関数などの外で宣言されたり、宣言がない変数はどこからでも参照が可
			 能な変数(\ElmJ{グローバル変数})になります。
 \item 関数内で宣言された変数はそれ以下の関数内で参照可能です。
\end{itemize}
関数の引数に対して、単純な変数は値渡し、配列は参照渡しとなります。
\begin{Verbatim}[numbers=left, fontsize=\small,
	commandchars=\\//,codes={\catcode`$=3\catcode`^=7}]
>foo =function(X,Y,Z) {
    X = X*2;
    Y[0] *= 2;
    Z.x *=2;}
function (X,Y,Z) {
    X = X*2;
    Y[0] *= 2;
    Z.x *=2;}
>A=1
1
>B=[2,3]
$\blacktriangleright$[2, 3]
>C={x:10,y:20};
$\blacktriangleright$Object {x: 10, y: 20}
>foo(A,B,C);
undefined
>A;
1
>B;
$\blacktriangleright$[4, 3]
>C;
$\blacktriangleright$Object {x: 20, y: 20}
\end{Verbatim}
\subsubsection{クロージャ}
いままでの\JS のプログラムでは、他の関数とデータのやり取りにグローバル変
数を多用してきました。グローバル変数を多用することは、複数人でプログラム
を開発するときなどに変数名の衝突がおき、それによってバグが発生する可能性
が増大します。これを避ける方法の一つとして%前節で示したような
関数の中で別の関数オブジェクトを作成する方法があります。
関数の中で新しく関数を定義すると、新しい関数ではその関数を定義する関数内
のローカル変数を参照できます。関数が実行されるときの環境を含めた状態を
\keyitem{クロージャ}と呼びます。

リスト\ref{svg-cycloid-animation-with-closure}はリスト
\ref{svg-cycloid-animation}(\pageref{svg-cycloid-animation}ページ)を改造
して、グローバル変数をまったく使わない
サイクロイドをアニメーションで描くものです。
\SVGListN{サイクロイドを描く --- アニメーション版(グローバル変数なし)}
{svg-cycloid-animation-with-closure3}
{svg-cycloid-animation-with-closure}
\begin{itemize}
 \item すべてのグローバル変数を\Event{onload}後に実行する関数内に移動しています
       (\LineR{GlobalS}{GlobalE})。
 \item その関数内でロカールな関数を定義しています
			 (\LineR{ClosureS}{ClosureE})。
\begin{itemize}
 \item 関数名は\Func{DrawNext}となっています(\Line{ClosureS})。
 \item \ElmJ{function}の前に\texttt{(}がついています。これに対応する
			 \texttt{)}は\Line{ClosureE}にあります。
			 %
  そのあとにある\texttt{()}はこの関数の引数リストです(ここでは
			 引数がないので空です)。
			 
  これにより関数を定義してその場で実行できます。
 \item 関数の定義位置が変わっただけで、残りのコードはほとんど同じです。
 \item 一番の違いは\Attrib{d}で必要となる座標データを配列として保存している
			 ことです(\Line{GlobalE})。
 \item また、\Line{push}で新たに計算した座標をこの配列に追加しています
			 (\ElmJ{push}メソッド)。
 \item \Line{join}の\ElmJ{join}は配列の要素を、引数の文字列をはさんでつ
       なげるメソッドです。配列内の文字列の連接
       を連接演算子\texttt{+}を用いるよりも簡単に記述できます。
 \item \Line{callee}ではこの関数自身がこの後、
			 呼び出せるように設定しています\footnote{関数の引数のリストを参照
			 する\ElmJ{arguments}のメンバー\ElmJ{callee}を用いると、無名関数を
			 参照できます。最新版の\JS では非推奨となっているので無名関数にし
			 ません。}。
\end{itemize}

\end{itemize}
\iffalse
	\subsection{オブジェクト指向\JS}
%\JS にはクラスという概念はありません。
\JS の連想配列を定義する右辺の表し
方は正式には\keyitem{オブジェクトリテラル}と呼ばれます。このキーに対する
値として関数を与えれば、あたかもメソッドのように取り扱うことができます。

図\ref{objectliteral}はリスト\ref{objectliteralL}のオブジェクトリテラル
の説明のための\JS のプログラムの実行結果です。

\ShowFig{0.7}{ht}{objectliteral}{オブジェクトリテラルの実行例}{objectliteral}
\HTMLListN{オブジェクトリテラルの実行例}{testObject1}{objectliteralL}

\begin{itemize}
 \item \LineR{ObjS}{ObjE}で変数\texttt{Point}にオブジェクトを定義してい
       ます。
 \item このオブジェクトには3つの要素が定義されています。これらの要素のう
       ち\Lines{x}{y}は数値を、\LineR{methodS}{methodE}では関数を与えて
       います。
 \item \texttt{setPos}では引数で指定された値をオブジェクトのメンバー
       \texttt{x}と\texttt{y}に設定しています。\ElmJ{this}は自分自身を指
       すオブジェクトです。
 \item \Lines{ShowX1}{ShowY1}では連想配列の形でメンバーの値を表示させて
       います。
 \item \Lines{ShowX2}{ShowY2}では通常のオブジェクトのメンバーを指定する
       方法を用いています。このほうがよりわかりやすい形になっていま
       すので、場面により使い分けることができます。\footnote{連想配列形
       式を利用すると\ElmJ{for}を用いてすべてのメンバーの情報を調べるこ
       とができます。これは情報が少ない、新しいオブジェクトを解析するとき
       に役立ちます。}
 \item \Line{ChangeProp}では\texttt{setPos}を用いてオブジェクトのメンバー
       の\texttt{x}と\texttt{y}の値を変更しています。実際に値が変更され
       ているのが確認できます。
 \item \Line{ChangeMethod}ではメソッド\texttt{setPos}を書き換えています。
       \JS はインタープリタ言語なので実行時に変更を行うことが可能です。
       \Line{CheckFunc}で結果を確認しています。       
\end{itemize}
ここで示した例では、同じようなオブジェクトを複数必要とする場合には同じ内
容を何度も書くことになります。これを避けるためには関数を定義してそれに
\ElmJ{new} キーワードを付けて呼び出すことで、オブジェクトのコンストラク
タとすることができます。図\ref{objectliteral2}がその例の実行結果です。
\ShowFig{0.7}{ht}{object2}{コンストラクタの実行例}{objectliteral2}
\HTMLListN{コンストラクタの実行例}{testObject2}{objectliteralL2}
\begin{itemize}
 \item \LineR{ObjS}{ObjE}で\texttt{Point}オブジェクトのコンストラクタを
       定義しています。このコンストラクタは2つの引数を持ちます。
\begin{itemize}
 \item \Lines{ChangeX}{ChangeY}でオブジェクトのメンバー\texttt{x}と
       \texttt{y}に引数の値を設定しています。
 \item \LineR{methodS}{methodE}でメソッド\texttt{setPos}を定義しています。
\end{itemize}
 \item \Line{New}で\texttt{Point}のインスタンスを作成し、変数\texttt{P1}に保存して
       います。
 \item \LineR{ShowX2}{ShowY4}でまえと同様に動作を確認しています。
 \item \Line{New2}で別のインスタンスを作成しています。
 \item \Line{ChangeMethod}で初めのインスタンスのメソッド\texttt{setPos}
       を変更しています。
 \item この変更が有効であることと、別のインスタンスには影響がないことを
       \LineR{CheckFunc}{ShowY5}で確認しています。
\end{itemize}
この例ではインスタンスのメソッドが同じコンストラクタから生成されたインス
タンスの間で共有されていないことを示します。同じコンストラクタから生成さ
れたインスタンスの間で共有したいデータはコンストラクタ関数の
\ElmJ{prototype}に定義します。

図\ref{objectliteral3}が\ElmJ{prototype}に共通のメソッドを定義した例です。
\ShowFig{0.7}{ht}{object3}{\texttt{prototype}の利用の実行例}{objectliteral3}
\HTMLListN{\texttt{prototype}の利用の実行例}{testObject3}{objectliteralL3}
このリストでは、\texttt{setPos}が\ElmJ{prototype}の関数として定義されて
います(\LineR{methodS}{methodE})。また、\Line{ChangeMethod}でメソッドを
変更し、それらのインスタンスすべてに変更が有効であることを確認しています
(\Lines{CheckFunc}{CheckFunc2})。
\subsection{オブジェクト指向の例}
図\ref{ShowSetClickPos}をオブジェクトを用いて書き直してみます。ただし、
ここでは複数の円を表示するので円の移動はクリックではなくドラッグで行いま
す。図\ref{interactive-object}がその画面です。
\ShowFig{0.9}{ht}{interactive-object}{オブジェクトを用いたHTML文
書}{interactive-object}

ここではSVG内の円と右側の円の中心位置の情報の表示を一つのオブジェクトと
して構成しています。リスト\ref{Point}がそのオブジェクトのリストです。こ
のファイル名は\texttt{Point.js}とします。
\JSListN{オブジェクトの定義}{Point}{Point}
\begin{itemize}
 \item 1行目から18行目で\texttt{Point}のコンストラクタ関数を定義していま
       す。
\begin{itemize}
 \item このコンストラクタは図形を表示するための親要素(\texttt{PFig})、中
       心位置を表示するための親要素(\texttt{PText})、円に関する情報の中心位置
       (\texttt{cx}と\texttt{cy})、塗りつぶしの色(\texttt{fill})と半径
       (\texttt{r})を引数に持ちます。
 \item このコンストラクタは生成された順番に番号を付けます(4行目)。そのた
       めに\ElmJ{prototype}に\texttt{No}というメンバーを定義しています
       (19行目)
 \item 7行目から8行目で図形の要素を定義してます。
 \item 9行目から16行目で円の中心位置を表示する要素を定義してそれぞれメン
       バーとして保存しています。
\end{itemize}
 \item 19行目では通し番号の初期化をしています。
 \item 20行目から24行目ではテキストボックスの値から円の位置を設定するメ
       ソッドを定義しています。
 \item 25行目から29行目では与えられた位置に円を移動させるメソッドを定義
       しています。この関数は主にドラッグ中に呼び出されます。
 \item 30行目から34行目では与えられた情報に基づいて表示を変更するメソッ
       ドを定義しています。
\end{itemize}
リスト\ref{interact-objectL}が図\ref{interactive-object}のリストです。
\HTMLListN{インターラクティブなオブジェク
ト}{interact-object}{interact-objectL}

このリストは\texttt{Point.js}以外にリスト7.14の\texttt{make-svg-elm.js}
も利用しています。
\begin{itemize}
 \item 9行目から40行目がで文書がロード完了後に呼び出される
関数を定義しています。ここでの\JS はこの関数だけです。
 \item 13行目から15行目で生成するインスタンスのリストを作成しています。
 \item 16行目から24行目でマウスボタンが押されたときの処理を定義していま
       す。ここでは現在設定されている円の中心位置とイベントが通知された
       ときの位置の差を保存していることに注意してください。
 \item 25行目から27行目ではマウスボタンが離されたときの処理を定義してい
       ます。
 \item 28行目から31行目ではドラッグ中の処理を定義してます。
 \item 32行目ではマウスボタンが押されたときと離されたときの処理をSVGの図
       形全体に登録しています。
 \item 33行目から39行目は、「設定」ボタンが押されたときの処理を定義して
       います。
\end{itemize}
\fi

\JS の配列のメソッドを用いて、\JS リス
ト\ref{pinna.js}を書き直したものが\JS リスト\ref{pinna-rev.js}です。

\JSListN{リスト\ref{pinna.html}で読み込む\JS ファイル--改良版}{pinna-rev}{pinna-rev.js}
\begin{itemize}
 \item \Line{Array.fill}で長さが6の配列に図形の構成要素の\Elm{path}を代
			 入しています。
			 \begin{itemize}
				\item \Verb+new Array(6)+は長さが6の配列が作成されますが、値は
							\ElmJ{undefined}で初期化されます。
				\item 値が\JSVal{undefined}の要素に対しては\ElmJA{map}などは適用され
							ません。
				\item したがって、\ElmJA{fill}で1に初期化し、その配列に対して
							\Elm{path}を代入しています。
			 \end{itemize}
 \item それぞれの\Elm{path}を構成するところにはそれぞれのパラメタを一つ
			 の配列にし、それらを集めて長さが6の配列にしたもの
			 (\Lines{forEachS}{forEachE})に対して\ElmJA{forEach}を適用していま
			 す。
 \item 前の例と同じにするために配列の成分を借り引数の名前と同じ変数に代
			 入しています(\Lines{paramSetS}{paramSetE})
\end{itemize}
\subsection{WebStorage}
Web ページ上で作成したデータをそのままブラウザが動作しているコンピュータ
上に保存することはセキュリティ上できません。しかし、ブラウザが管理する領
域に保存することは可能です。この節ではそれらの方法につい
て簡単な解説をします。

\ElmJ{Web Storage}とは HTML5 で導入された、Webページ間のデータの共有する仕組みで
す。HTML5以前の方法ではクッキーと呼ばれるデータをクライアントとサーバー
の間で共有してサーバー側からデータの共有をサポートする仕組みがありました。
クッキーでは小さなデータしか扱えず、サーバーとクライアントとのやりがある
たびに送られるのでセキュリティや速度の面で問題がありました。

HTML5ではこれに代わる方法としてブラウザーのあるページ以降のページにだけ
存在する\ElmJ{sessionStorage}とページが閉じられてもデータが存続できる
\ElmJ{localStorage}が提案されています。なお、これらの値はすべて文字列で
しか保存できません。

リスト\ref{pinna-storage.js}はリスト
\ref{pinna.html}(\pageref{pinna.html}ページ)
において最後に設定した色をStorageに保存しておくように\JS ファイルを書き
直したものです。
\JSListN{Storageを利用する}{pinna-storage}{pinna-storage.js}
このリストは\texttt{Storage}に対応する部分が追加されているだけです。
\begin{itemize}
 \item \Line{storage}で変数\texttt{Storage}に
       \texttt{window.localStorage}を代入しています。これは後で
       \ElmJ{localStorage}に書き直すのを簡単にするためです。
 \item \Line{setFromStorage1}ではテキストボックス「色1」の初期値を
       \texttt{Storage.C1}が定義されていたらその値を、そうでない場合には
       \texttt{''red''}に設定しています。
 \item 「色2」の方も同様です。
 \item 「設定」ボタンが押されたときに呼び出される関数\texttt{DrawFigs}で
       は色の情報を対応する\texttt{Storage}に格納しています
       (\Lines{saveFromStorage1}{saveFromStorage2})。
\end{itemize}