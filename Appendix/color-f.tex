% -*- coding: utf-8 -*-
\newcommand{\G}[4]{%
%   \raisebox{-1.5mm}{\includegraphics[width=0.5cm]{Appendix/#1.eps}}
   \raisebox{-1mm}{\rule[-0.5mm]{0mm}{5.mm}%
    \fcolorbox[rgb]{0,0,0}{#4}{\rule{4mm}{0cm}\rule{0cm}{4mm}}}
   &#1
   &\TMP #2 &\texttt{#3}\\\hline}
\newcommand{\TBox}[1]{\makebox[2em][r]{#1}}
\def\TMP(#1,#2,#3){\texttt{rgb(\TBox{#1},\TBox{#2},\TBox{#3})}}
\newcommand{\RGB}[2]{}
\setlength{\fboxrule}{0.1mm}
\setlength{\fboxsep}{0mm}
\chapter{SVGで利用できる色名}\label{SVGColor}\setcounter{page}{1}\vspace*{-1em}
次の表はSVGで利用できる色名とその\texttt{rbg}値の一覧表
です\cite{SVG11}\footnote{
\texttt{\#} で始まる3桁の16進表示は個々の数値を繰り返した16進表示と同じです。たと
えば\protect\texttt{\#fb8}は\texttt{\#ffbb88}を意味します%
(\href{https://www.w3.org/TR/2011/REC-SVG11-20110816/types.html\#DataTypeColor}
{https://www.w3.org/TR/2011/REC-SVG11-20110816/types.html\#DataTypeColor}
)。}。
{\small
\begin{longtable}{|c|c|c|c|}
\hline
色&色名&rgb値&16進表示\\\hline
\endfirsthead
\hline
色&色名&rgb値&16進表示\\\hline
\endhead
\multicolumn{4}{r}{次のページへ}
\endfoot
\hline
\endlastfoot
\input Appendix/color-fff.tex
\end{longtable}
}