% -*- coding: utf-8 -*-
\iffalse
\chapter{\thechapter{} Webページのデバッグ方法}
\section{最近のブラウザ上の開発環境}
最近のブラウザーはいろいろなデバッグ環境を取り揃えています。
その機能には次のようなものがあります。
\begin{itemize}
 \item XML文書の構造(DOMツリー)の表示と属性値の直接の編集
 \item CSSの表示と編集
 \item \JS のデバッグの機能

通常のデバッグ機能であるブレークポイントやステップ実行などができます。
\JS にエラーがあるとその位置を教えてくれます。
 \item コンソールによる\JS の直接の実行、途中結果の表示(以前は
       \ElmJ{alert}による表示しかできませんでした。)
 \item ネットワークの利用状況(ファイルがどのような順番でダウンロードされ
       たかをグラフィカルに表示し、ボトルネックの解明ができます)。
 \item \keyitem{クッキー}や \keyitem{WebStorage}の管理と表示
\end{itemize}
この章ではメジャーなブラウザにおいてこの機能を利用する方法を簡単に説明します。
\section{\Operan}
\Operan におけるデバッグツールは\keyitem{Opera Dragonfly}です。
これを開始するためにはメニューから
\begin{center}
 「ツール」$\longrightarrow$「詳細ツール」$\longrightarrow$「Opera Dragonfly」
\end{center}
を選択します(図\ref{operaDragonflyStart})
\ShowFig{0.95}{ht}{operaDragonflyStart}
{Opera Dragonfly の開始方法}{operaDragonflyStart}

このときウィンドウの下部に Opera Dragonfly 
が現れます(図\ref{operaDragonflyStartAfter})。
\ShowFig{0.95}{ht}{operaDragonflyStartAfter}
{Opera Dragonfly の開始直後の画面}{operaDragonflyStartAfter}

この上部には Opera Dragonfly のメニューが現れます。この図では「ドキュメ
ント」が選択されているので DOM ツリーが表示されています。DOMツリーは展開
したりまとめたりできます。それぞれのタブを選択することで上に述べた機能を
使用できます。
\section{\IE 9}
\fi