%-*- coding: utf-8 -*-
\chapter{SVG入門}
\section{SVGファイルを作成する方法と作成上の注意}
SVGはテキストベースの画像表現フォーマットです。
%\footnote{テキストベースといっても顔文字とは違います。}
これを作成するためには
\keyitem{エディタ}と呼ばれるソフトウェアを使います。Windows で標準につい
てくるメモ帳はエディタの例です。また、XML 専用のエディタ
を利用するのもよいでしょう。
%XML の文書を編集するためのエディタも便利です。
\begin{Problem}
エディタソフトウェアにはどのようなものがあるか調べなさい。特に Unix では
 どのようなものが有名であるか調べること。また、XML文書を編集するためのエ
 ディターにどのようなものがあるか調べなさい。
\end{Problem}
エディターでSVGファイルを作成するときの注意を次にあげておきます。
\begin{itemize}
 \item SVGファイルの内容はHTMLファイルのようにタグをつけた形で記述します。
       タグで定義されるものを要素と呼びます。
 \item HTMLファイルのタグでは次のようなことが可能です。
\begin{itemize}
 \item 要素名は大文字小文字の区別がなく, 要素の開始を表す部分と終了を表す
       部分で大文字小文字が違っていても問題はおきません。
 \item 改行を示す\ElmH{BR}のように終了部分がない要素もあります。
\end{itemize}
 \item SVGファイルは厳
       密なXMLの構文に従うので要素の開始部分に対応する対応する終了部分を必ず記
       述する必要があります(簡略な形式もあります)。また、大文字小文字も
       完全に一致している必要があります。うまく動かないときにはこの点を
       確かめましょう。\footnote{厳密なXMLの構文に従うHTMLの文書としては
       XHTMLの形式があります。この形式では\ElmH{BR}の代わりに
       \ShowElm{BR/} と記述することで終了タグが不要になります。}
 \item 要素などに記述する単語は英語です。複数形を示す\texttt{s}がついて
       いるのを忘れたりするだけで動かなくなるのでよくチェックしましょう。
 \item ファイルの拡張子は \texttt{svg} が標準です。
\end{itemize}

このテキストではいろいろなSVGファイルのリストが出てきます。リストの入力
に対しては次のことに注意してください。
\begin{itemize}
 \item リストの各行の先頭には行番号が書いてありますが, これは
       説明のためにつけたものなので, エディタで入力するときは必要
       ありません。
 \item 要素の間の空白は原則的にはひとつ以上あれば十分です。見やすく読み
       やすくなるようにきれいに記述しましょう。
\end{itemize}

\Opera におけるXML文書としてのエラーがあった場合の表示例を解説します。
\SVGListN{XML文書としてエラーがあった場合}{first}{first}

リスト\ref{first}では6行目で
\texttt{x="0" y="0"} と記すべきところを \texttt{x="0"y="0"} と空白を入れ
       なかった場合のエラーの表示です。

\Opera{} では
エラーがあった位置を行数とその行からの位置でに指摘してくれます。
\ShowFig{0.6}{ht}{error-1}{\Opera におけるエラーメッセージ}
{error-opera-1}

これでよくわからない場合には次のような方法もあります。
\begin{enumerate}
 \item 図\ref{error-opera-1}で右クリックすると図\ref{error-opera-2}のコ
       ンテキストメニューが表示されます。
\ShowFig{0.6}{ht}{error-2}{エラー画面でのコンテキストメニューの表
       示}
{error-opera-2}
 \item ここで「ソースの検証」をクリックすると図\ref{error-opera-3}のメッ
       セージボックスが表示されます。
\ShowFig{0.6}{ht}{error-3}{ファイルのアップロードの確認のメッセー
       ジボックス}
{error-opera-3}
 \item ここで「はい」のボタンを押すと W3C にデータが送られてデータの検証
       が行われます(図\ref{error-opera-4})。
\ShowFig{0.6}{ht}{error-4}{検証結果の表示画面}
{error-opera-4}
 \item このページを下のほうにスクロールすると間違いの理由と場所がより具
       体的に指摘されます(図\ref{error-opera-5})。
\ShowFig{0.6}{ht}{error-5}{検証結果の表示画面(2)}
{error-opera-5}
\end{enumerate}
\begin{Problem}\upshape
 \IEn、 \FFn や Chrome でエラー表示がどのようになるかチェックしなさい。
\end{Problem}

\input CH2/0201svgfirst.tex
\input CH2/0202circle.tex
\input CH2/0230gradiation.tex
\input CH2/0240opacity.tex

%\section{HTML文書内に\SVG を取り込む}

